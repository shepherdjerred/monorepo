\documentclass{mla}

% Smart quotes
\usepackage{csquotes}
\MakeOuterQuote{"}

% Minimum of 8 pages
% 2200-2500 words, not including words cited
% Minimum of 8 sources, 3 of which are books

% Bibliography files
\addbibresource{bibliography.bib}

% Show misc bibliography entries
\DeclareBibliographyAlias{misc}{article}

% Header
\firstname{Jerred}
\lastname{Shepherd}
\professor{Dr. Brown}
\class{ENG 211-04}
\title{The Case for Automation}

\begin{document}
\makeheader

Since their invention, cars have become a crucial part of modern-day society. Despite being only a couple of centuries old, they have allowed transportation to be both more accessible and much quicker. Cars allow for populations to be more spread out  while still being within reach. Self-driving cars are the next iteration of this technology and will spur a similar transportation revolution due to their numerous benefits. Self-driving cars will eliminate the menial task of driving, make automobiles a safer form of transportation, and shift today's culture in many ways.

Some may have doubts about the practicality of self-driving cars, or they may be even unaware of their existence. After all, self-driving cars do sound like a far-off invention but they do actually exist today despite their Jetson's-sounding name. They are not some abstract concept that has been yet to be realized. In fact, companies have been testing their self-driving cars on public roads for several years now. The Defense Advanced Research Projects Agency, or DARPA, hosted an event called the DARPA Grand Challenge in 2004, which spurred the development of autonomous vehicles \cite{DARPA2014}. According to DARPA, researchers were tasked to navigate a 142-mile stretch of desert. Of the fifteen teams that competed in the challenge, the furthest went only 7.5 miles. DARPA hosted a second event only eighteen months later. During this second event, five teams out of 195 were able to complete a 132-mile course. This is incredibly impressive considering that the teams were able to improve so much in such a short period of time.

% TODO fix multiple citation style
The field of autonomous driving has made progress at a very rapid pace, and part of this is because of the many tech companies and Silicon Valley startups who have the goal of developing a fully-autonomous vehicle. One example is Waymo, a subsidiary of Google, who has driven over ten million miles in California with their fleet of 600 self-driving cars in the last ten years (Hawkins; Mayersohn). Waymo also offers a public self-driving taxi service in Arizona, which is a real-world application of the autonomous vehicle technology. Tesla, a high-end electric car company based in California, currently sells cars that have the hardware required for full self-driving capabilities. The amount of progress made towards fully autonomous vehicles is amazing considering its recent origins.

% TODO cite
A critical aspect of autonomous vehicles is machine vision which is the process of recognizing objects in an image \cite{Goodfellow2016}. This technology is so important because it allows the onboard computer to know where objects such as lanes, cars, and pedestrians are. Convolutional neural networks, which essentially teaches computers to recognize objects in a set of training images, are an approach to machine vision that became popular in the early 2000's. Once the computer is trained, it can recognize objects in live data such as images obtained from sensors mounted on a vehicle.

% TODO better transition
Although fully self-driving cars are not yet commercially available, all cars built today leverage the computers to assist human drivers. Computers for decades have been embedded into cars to control braking, safety features, and engine internals \cite{NHTSA2019}. Modern car computers have started taking on more responsibility by more directly, especially in the realm of safety features. Some of these advanced features include automated lane switching, adaptive cruise control, and self-parking.

SAE International, an organization of engineers, defined six levels of driving automation. These six levels have become a standard way to measure the degree of autonomy of a vehicle \cite{SAEInternational2018}. These six levels range from no automation at level 0, to full automation at level 5. The four levels in-between represent a middle-ground and most cars made today fall in the lower half of this range. Audi released the first consumer vehicle with level 3 automation features, the Audi A8, in 2019 \cite{Davies2018}. Level 3 automation allows a vehicle to drive itself until the vehicle prompts the human operator to take over. This is similar to level 2 automation, except level 2 automation requires the driver to stay alert while the vehicle drives whereas level 3 does not. While these features do not quite reach the goal of fully automated driving, they do demonstrate the ceding of control from human drivers to computers.

% TODO revise
% TODO better transition
Driving a car is inherently dangerous. Asking a person to operate a vehicle traveling at high speeds is a dangerous task no matter how well prepared or trained the person is. Driving safely imposes a host of requirements and rules on all drivers, and when these requirements are not met or rules not followed, accidents occur. Drivers are expected to know and obey all traffic laws. This alone is a monumental task considering how numerous and complex the laws are, but the requirements don't stop there. Human drivers are expected to give their full focus to the task of driving at all times; any lapse in their attention while driving can prove fatal both to themselves or others. Again, this is a huge burden to put on drivers but it cannot be avoided. Those who drive professionally may drive for up to a 14-hour stretch, and it is very doubtful that they are able to keep out all distractions \cite{FMCSA2015}. Some may be too exhausted to drive fall asleep at the wheel, which is another limitation of human drivers. But even if humans were able to meet the previous requirements it still would not be enough. Humans are limited in both their speed and in their judgement. In situations where seconds matter, drivers simply do not have the time to think through their options. Instead, they rely on mostly instinct. If they make the wrong choice, it could lead to a fatal accident that may of been avoidable.

% TODO revise
% TODO find more figures to support this
Safety has always been a point of concern with cars. Government agencies exist to ensure that cars meet standards to ensure that they are safe, and car manufacturers often advertise how safe their cars are to customers. Those creating self-driving cars have a tougher job when trying to convince others that their vehicles are safe. Not only do they have to take traditional design considerations into account, but they must also ensure that their car drives itself in a safe manner. A study done in 2017 showed that Google's self-driving cars have a lower crash involvement rate compared to human drivers \cite{Teoh2017}.

% TODO revise
% TODO better transition
Computers will be better suited to the task of driving once they have been sufficiently programmed. They are able to perform billions of calculations a second, and their judgement does not falter during stressful situations. They do not have the trouble of remembering a myriad of traffic laws after they are added, and they do not fatigue while driving for a long period of time. Self-driving cars have the potential to address the problems of human drivers, and make roads much safer.

% TODO revise
Self-driving cars may be safer than those piloted by humans, but there will inevitably be accidents. No system is perfect, no matter how well designed it is. There may be some ambiguous situation that a vehicle runs into, or a hardware failure of some sort. Some people are uncomfortable with the idea of giving away control of their vehicle to a computer --- they feel that even if statistically an autonomous vehicle were significantly safer than themselves, they would rather drive themself. This response is understandable and it stems from a distrust in technology, not in giving up control. People willingly give up control when flying as a passenger in an airplane, or even when riding as a passenger in a car while someone else drives. Considering that these computers are statistically less likely to end up in an accident compared to human drivers, there is no rational reason to distrust an autonomous vehicle any more than you would distrust your friend driving. It's doubtful that traffic accidents will ever be completely eliminated, but a dent can certainly be made.

% TODO revise
The benefits of self-driving cars are not confined to safety. Self-driving cars have the potential to revolutionize not only transportation, but society as a whole. Humanity has seen many of these revolutions, such as the inventions of trains, cars, and planes, all of which have rapidly decreased travel time. While self-driving cars wouldn't necessarily decrease travel time, it would allow those who would be driving to focus on other tasks. Instead of spending their morning commute on a menial task, they have the opportunity to focus on other things. Those who are impaired or underage and could not otherwise drive a car would be able to gain independence and travel on their own. Commutes would be less of a bother, which would allow humans to live further from cities while still being able to easily access them \cite{Zakharenko2016}.

% TODO revise
With cars being able to drive themselves, parking becomes less of an issue \cite{Zakharenko2016}. People may be able to drive to their destination, and allow their car to park somewhere far off until they need it again, where it will then drive back to the owner. Economic centers will be able to devote less space to parking, allowing for these centers to be more dense due to the space that may be reclaimed. This is especially helpful in the downtown areas of major cities, where parking can come at premium.

% TODO revise; this is awkward
Over 10 million Americans were employed in the transportation industry in 2016 \cite{BTS2018}. Employees in this sector all perform the task of transporting some object between two points, which is a task that can easily be done with a self-driving vehicle. The development and adoption of self-driving vehicles would allow the jobs in this sector to be eliminated, which would allow for these workers to focus on other more interesting tasks. While the elimination of jobs is not positive in the short-term, it is necessary for the progression of society. The majority of experts in the field of AI and robotics canvassed in a study by the Pew Research Center believe that more jobs will be created than eliminated \cite{Smith2014}. While there may be a period of increase unemployment, it will even out as time goes on.

% Enviornment
The environment may also have something to gain with the adoption of self-driving cars. If self-driving cars become safe enough, they may be able to strip safety features which add a significant amount of weight to the vehicles \cite{Worland2016}. With less weight comes less fuel consumption. Self-driving cars could also encourage ride-sharing and reduce the need for car ownership. Considering that large number of people in the United States have no need for their car late at night or during work hours, these car owners may be able to rent their car out while they sleep or work. This would allow others to have access to transportation without needing a car of their own. A figure by the Department of Energy claims that energy consumption due to transportation could be reduced by 90\%, or increased by 250\% depending on factors such as the number of people being transported, the number of cars on the road, and the distance travelled. While increased energy usage is not good, it can be changed through legislation. If laws were passed that encourage or require carpooling for owners of self-driving cars, then energy usage can be reduced.

% TODO revise
One roadblocks when it comes to self-driving cars is the legal ambiguity that surrounds it. In fact, the CEO of Volvo believes that this challenge will be harder to overcome than the technology required for the vehicles \cite{Brodsky2016}. While self-driving cars are probably legal in the United States according to Brodsky, there is an unresolved question of liability. Brodsky states that "if society applies existing tort and contract law to autonomous vehicles, liability for accidents will rest either with the manufacturer of the vehicle or with the driver." This question may be enough to deter manufacturers from commercially releasing automated vehicles until it is answered. In fact, this has already led to concrete action by Audi, a German car manufacturer. As mentioned previously, the Audi A8 was the first vehicle released with level 3 driving automation with a feature called Traffic Jam Pilot \cite{Davies2018}. This feature allows the car to drive itself without any human supervision while traveling at a speed below thirty-six miles per hour. According to Audi this feature is unavailable in the U.S. due to legal concerns.

% TODO revise
While self-driving cars are not yet available to consumers, the technology has advanced rapidly and will have a far-reaching impact once adopted. Modern consumer cars commonly have computer-assisted driving features, but none offer true automation. Many companies are developing and testing fully autonomous vehicles today. Self-driving cars have a host of benefits, including improved safety and accessibility. Self-driving cars also have the potential to shift the culture of the nation by allowing people to live further from cities while still being able to easily access them. There are both legal and technological hurdles that must be overcome before these vehicles become widely available. Once these obstacles have been passed, humans will overall benefit from the technology.

\makeworkscited

\end{document}
